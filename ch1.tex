\chapter{Introduction to thesis writing}
\label{chap:refs}

\section{So you have to do a thesis?}

% this section is translated from https://e-x-a.org/rp/
% (thanks go to Jiri Benes for the initial translation)

If you are reading this, you likely have to write a thesis to finish your Bachelor or Master studies. First, congratulations on getting this far! What is a thesis though?

A thesis is a monograph, a longer writing about a given single topic of choice. The topic is typically implied by your studies --- at the school of computer science, you will write about solving a problem with computers; at medical school you will write about an issue about health or anatomy, etc. To create the thesis, you typically proceed as follows:
\begin{enumerate}
\item You conduct a moderate review of current (or recent) \emph{academic literature} on a specific topic,
\item with the acquired knowledge, you solve a narrowly specialized and \emph{well-defined problem},
\item through standard and \emph{reproducible methods}, you determine the quality of your solution, and finally\item you express the entire story \emph{comprehensibly}, clearly, and lucidly in a 20- to 60-page book, which becomes your thesis.
\end{enumerate}

By submitting a thesis, the student proves that they are capable of understanding the results of recent research, using them to scientifically guide and analyse their own work, and communicating the outcome comprehensibly to others.

The work on the thesis is typically started by finding a supervisor and agreeing on the topic. You will typically need to write an annotation of the thesis, which comprises a condensed summary of the main idea and expected results. The choices are submitted to administration, and after some time, you will be officially tasked with completing the thesis assignment.

Before the thesis is assigned, you typically prepare for the main work; e.g., by testing all the software libraries, tools, and methods that you will need, exploring your target domain, environment and subjects, creating and testing various prototypes and proof-of-concepts, reading relevant academic literature, and refining your approach (and the thesis annotation) based on initial findings.

The preparation phase is often quite crucial, as it is where you may discover many potential roadblocks and refine your approach before committing to the full thesis. Generally, the work conducted in advance gives you much better chance to finish the thesis without surprises later. For similar reasons, one of the main tasks of the supervisor is to know the roadblocks that you do not see yet, and prevent you from committing to a thesis that is not finishable or defendable.

\subsection{How to recognize a good thesis topic?}

Before you officially register a thesis --- and ideally before you start committing to the task at all --- you should know the concrete instances of terms highlighted in the points 1--4 above. In particular, you should be able to clearly answer the following questions:

\begin{description}
\item[Is my idea even relevant?]
To get the answer, go to Google Scholar and try to find articles that are not too old (say, at most 15--20 years, depending on the area), are not visibly outdated, and are relevant to your topic. If you cannot find any recent academic work on your topic, it might be too niche, already solved, or not academically interesting. Your supervisor will often help you find the latest research in the area.
\item[Is the problem I want to solve suitable?]
No one expects you to produce a completely new and revolutionary solution to your problem. The method for solving your problem should be known and already explored --- in particular, you are not expected to make any groundbreaking scientific discoveries! You should ``just'' properly use the existing methods in a slightly novel context of your problem. For example, you may apply a known approach in a way that no one has tried before, slightly modify the approach to enable use in a seemingly different domain, or combine existing approaches in a novel way. In other words: the problem should be non-trivial, but you should know in advance how to approach the solution.
\item[Is the problem well-defined?]
Solve one problem, and solve it well. You will often encounter a thesis reviewer telling you that your problem is not really a problem --- this is especially common for theses with topics such as ``I'll make a video game about X''. To provide an answer, you must have a direct, undeniable proof that your problem is relevant and its solution constitutes at least a minor advancement in some area (ideally, the world will be much more beautiful and civilization will be better off with the problem solved). Ideally, you will be able to point out existing academic literature that states that your topic is indeed a problem worth solving.
\item[How will I explain the problem to others?]
Write your thesis proposal as a hypothesis, and design an experiment that you can conduct to either confirm or reject this hypothesis, or to clearly determine the boundaries of its validity. Some theses are not experiments per se, but may be easily reworded as a case study, user testing, performance analysis, or theoretical proof. Mainly, the hypothesis--experiment wording helps you to convey the difficulty of the task, and clearly point out the ``unknowns'' that you aim to uncover.
\item[Am I trying to solve an exceedingly hard problem, or multiple problems at once?]
In such cases, you might be shooting yourself in the foot: You will lose a lot of time on something that is not necessary for the thesis, and your thesis reviewer will get annoyed trying to spot the single problem that you are solving. If your thesis addresses multiple smaller problems, you must correctly frame them in a single, well-defined and topically consistent group of problems.
\item[How do I present my results?]
In your thesis, you will have to prove that you actually solved something, and reviewers will, by default, actively doubt that you accomplished anything meaningful. Prepare a clear method for evaluating your results. In computer science, the most common evaluation methods include performance measurements (speed, memory usage, accuracy), comparative analysis (how does your solution compare to existing approaches?), case studies (real-world applications of your solution), user studies (how do people interact with and benefit from your work?), and theoretical analysis (including mathematical proofs and logical arguments). It is extremely useful to show plots, statistics, and various other visualized evidence.
\item[How do I interpret my results?]
Ideally, you will compare your results with ones from people who solved similar problems. Besides measuring your own solution, you might also need to measure the alternative implementations and approaches, use a comparable measurement methodology (benchmark) to allow comparison to other benchmarked work, and ensure fair comparison conditions. \emph{However, no one expects you to win such comparison!} A negative result is also a result, and thesis reviewers understand this. On the other hand, they will have no understanding for irreproducible measurements or unfair comparisons.
\item[How will I fill all these pages with text?]
Can you write 20--60 moderately amusing pages about your work? If not, you will be able to gain inspiration from reading academic literature thoroughly and summarizing your findings (see above), conducting your own experiments and analysis, reflecting on the broader implications of your work. Typically, you can write a lot of good text simply by reviewing 2--3 research papers that your thesis builds upon, and summarizing them within the context of your problem.
\item[Am I doing it right?]
To understand what is typically expected, examine some of the completed theses in your field. Look into thesis repositories from your institution\footnote{\url{https://dspace.cuni.cz/} for Charles University}, examine what makes a thesis successful, observe the different approaches to similar problems, writing styles and organizational structures. To get a good picture of the common issues and expectations, examine the official reviews from the supervisors and reviewers on several theses (these are typically deposited along with the thesis).
\item[Am I making it in time?]
Allocate a lot of time for writing, editing, and making good pictures. At the \emph{extreme barest minimum}, allocate at least one full month of work for fixing and incorporating the comments from your supervisor, especially if you never wrote a thesis before. Check the thesis submission requirements in advance; in particular, if you are supposed to print your thesis on paper, order the printing service with sufficient time reserve.
\item[Who will tell me what to do next?]
You will. Remember that it is \emph{your} thesis --- while your supervisor will try to guide you, it is ultimately you who is responsible for your own work (and success). You are thus expected to take initiative (and many would argue that the main point of the theses is precisely to \emph{show the initiative}). In particular, never wait for your advisor to give you tasks. Be proactive, work independently, solve issues well before deadlines, come to meetings equipped with prepared questions and clearly summarized progress updates and milestones, and bring up your own ideas about next steps.
\end{description}

\section{What to put into the first chapter?}

First chapter usually builds the theoretical background necessary for readers to understand the rest of the thesis. You should summarize and reference a lot of existing literature and research.

You should use the standard \emph{citations}\todo{Use \textbackslash{}emph command like this, to highlight the first occurrence of an important word or term. Reader will notice it, and hopefully remember the importance.}.

\begin{description}
\item[Obtaining bibTeX citation] Go to Google Scholar\footnote{\url{https://scholar.google.com}}\todo{This footnote is an acceptable way to `cite' webpages or URLs. Documents without proper titles, authors and publishers generally do not form citations. For this reason, avoid citations of wikipedia pages.}, find the relevant literature, click the tiny double-quote button below the link, and copy the bibTeX entry.
\item[Saving the citation] Insert the bibTeX entry to the file \texttt{refs.bib}. On the first line of the entry you should see the short reference name --- from Scholar, it usually looks like \texttt{author2015title} --- you will use that to refer to the citation.
\item[Using the citation] Use the \verb|\cite| command to typeset the citation number correctly in the text; a long citation description will be automatically added to the bibliography at the end of the thesis. Always use a non-breakable space before the citing parenthesis to avoid unacceptable line breaks:
\begin{Verbatim}
Trees utilize gravity to invade ye
noble sires~\cite{newton1666apple}.
\end{Verbatim}
\item[Why should I bother with citations at all?] For two main reasons:
\begin{itemize}
\item You do not have to explain everything in the thesis; instead you send the reader to refer to details in some other literature. Use citations to simplify the detailed explanations.
\item If you describe something that already exists without using a citation, the reviewer may think that you \emph{claim} to have invented it. Expectably, they will demand academic correctness, and, from your perspective, being accused of plagiarism is not a good starting point for a successful defense. Use citations to identify the people who invented the ideas that you build upon.
\end{itemize}
\item[How many citations should I use?]
Cite any non-trivial building block or assumption that you use, if it is published in the literature. You do not have to cite trivia, such as the basic definitions taught in the introductory courses.

The rule of thumb is that you should read, understand and briefly review at least around 4 scientific papers. A thesis that contains less than 3 sound citations will spark doubt in reviewers.
\end{description}

There are several main commands for inserting citations, used as follows:
\begin{itemize}
\item \citet{knuth1979tex} described a great system for typesetting theses.
\item We are typesetting this thesis with \LaTeX, which is based on \TeX{} and METAFONT~\cite{knuth1979tex}.
\item \TeX{} was expanded to \LaTeX{} by \citet{lamport1994latex}, hence the name.
\item Revered are the authors of these systems!~\cite{knuth1979tex,lamport1994latex}
\end{itemize}

\section{Some extra assorted hints before you start writing English}

\paragraph{Word order}
Strictly adhere to the English word order rules. The sentences follow a fixed structure with a subject followed by a verb and an object (in this order). Exceptions to this rule must be handled specially, and usually separated by commas.

\paragraph{Sentence structure}
Do not write long sentences. One sentence should contain exactly one fact. Multiple facts should be grouped in a paragraph to communicate one coherent idea. Both the sentences and paragraphs should include various hints about their relation to the other ideas and paragraps. These are typically materialized as adverbs or short sentence parts that clarify the cause--outcome and target--method--result relationship of the sentences in a paragraph. Such `word glue' helps the readers to correctly draw the lines that hold their mental images of your thesis together, and ideally see the big picture of what you were trying to convey right from the first read.

Paragraphs are grouped in labeled sections for a sole purpose of making the navigation in the thesis easier. Do not use the headings as `names for paragraphs' --- the text should make perfect sense even if all headings are removed. If a section of your text contains one paragraph per heading, you might have wanted to write an explicit list instead.

Mind the rules for placing commas:
\begin{itemize}
\item Do not use the comma before subordinate clauses that begin with `that' (like this one). English does not use subordinate clauses as often as Slavic languages because the lack of a suitable word inflection method makes them hard to understand. In scientific English, try to avoid them as much as possible. Ask doubtfully whether each `which' and `when' is necessary --- most of these helper conjunctions can be removed by converting the clause to non-subordinate.

As an usual example, \xxx{\textit{`The sentence, which I wrote, seemed ugly.'}} is perfectly bad; slightly improved by \xxx{\textit{`The sentence that I wrote seemed ugly.'}}, which can be easily reduced to \textit{`The sentence I wrote seemed ugly.'}. A final version with added storytelling value could say \textit{`I wrote a sentence but it seemed ugly.'}
\item Use the \emph{Oxford comma} before `and' and `or' at the end of a longer, comma-separated list of items. Certainly use it to disambiguate any possible mixtures of conjunctions: \textit{`The car is available in red, red and green, and green versions.'} Remember that English `or' is typically understood more like `either this or that, but not both,' and the use of `and` is much more appropriate in cases such as possibility overviews and example listings (like in this sentence).
\item Consider placing extra commas around any parts of the sentence that break the usual word order, especially if they are longer than a single word.
\end{itemize}

\paragraph{Nouns}
Every noun needs a determiner (`a', `the', `my', `some', \dots); the exceptions to this rule, such as non-adjectivized names and indeterminate plural, are relatively scarce. Without a determiner, a noun can be easily mistaken for something completely different, such as an adjective or a verb.

Name all things with appropriate nouns to help both the reader and yourself, and do not hesitate to invent good names and labels for anything that you will refer to more than once. Proper naming will save you a lot of writing effort because you will not have to repeat descriptions such as \xxx{\textit{`the third output of the second benchmarked method of the improved set,'}} instead you may introduce a labeling that will allow you to say just something like \textit{`output M2\textsuperscript{+}-3'}. At the same time, this will reduce the risk that the reader will confuse the object with another one --- for illustration, the long version of the previous example might very easily confuse with the second output of the third method. The same also applies to methods descriptions, algorithms, programs, testing datasets, theorems, use-cases, challenges and other things. As an example, \xxx{\textit{`the algorithm that organizes the potatoes into appropriate buckets'}} shortens nicely as \textit{`the potato bucketer'} and may be labeled as a procedure \textsc{BucketPotatoes()}, and \xxx{\textit{`the issue where the robot crashes into a wall and takes significant time to return to the previous task'}} may be called just \textit{`the crash--recovery lag'}.

\paragraph{Verbs}
Although English can express a whopping 65 base verb tenses and their variants, scientific literature often suppresses this complexity and uses only several basic tenses where the meaning is clearly defined. Typically, you state facts in present simple (\textit{`Theorem 1 proves that Gadget B works as intended.'}), talk about previous work and experiments done in past simple (\textit{`We constructed Gadget B from Gizmo C, which was previously prepared by Tinkerer et al.'}), and identify achieved results in present perfect (\textit{`We have constructed Technology T.'}). Avoid using future tense, except for sections that explicitly describe future work --- as a typical mistake, if you state that the thesis \emph{will} describe something in later chapters, you imply that the description is not present there yet.

Do not write sentences in passive voice, unless you explicitly need to highlight that something has passively subjected itself to an action. Active voice is more preferable in the theses because it clearly highlights the actors and their contributions --- typically, \textit{`you did it'} instead of \textit{`it was done'} by a mysterious entity, which the reviewers rarely envision as yourself. Writing in active voice additionally benefits the explanation of complex processes: There, the word order forces you to identify the acting subject as the first word in the sentence, which further disambiguates how the individual process parts are triggered and ordered.

Try to avoid overusing gerunds (verbs that end with `-ing'). It is convenient to write shorter sentences by using gerunds as adjectives, but these are typically quite hard to understand because the readers may easily confuse the intended adjectives with verbs. If your sentence contains two gerunds close to each other, it may need a rewrite.

\paragraph{Scientific writing resources}
Consult the book by \citet{glasman2010science} for more useful details and recommended terminology for writing about the scientific research. Very pragmatically, the book by \citet{sparling1989english} describes many common mistakes that Czech and Slovak (and generally Slavic) writers make when writing English.
